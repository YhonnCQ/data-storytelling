\documentclass[twoside,twocolumn]{article}

\usepackage{blindtext} 
\usepackage{graphicx}
\usepackage[sc]{mathpazo} 
\usepackage[T1]{fontenc} 
\linespread{1.05} 
\usepackage{microtype} 


\usepackage[spanish,english]{babel} 


\usepackage[hmarginratio=1:1,top=32mm,columnsep=20pt]{geometry} 
\usepackage[hang, small,labelfont=bf,up,textfont=it,up]{caption} 
\usepackage{booktabs} 


\usepackage{lettrine} 


\usepackage{enumitem} 
\setlist[itemize]{noitemsep} 


\usepackage{abstract} 
\renewcommand{\abstractnamefont}{\normalfont\bfseries} 
\renewcommand{\abstracttextfont}{\normalfont\small\itshape} 


\usepackage{titlesec} 
\renewcommand\thesection{\Roman{section}} % 
\renewcommand\thesubsection{\roman{subsection}} 
\titleformat{\section}[block]{\large\scshape\centering}{\thesection.}{1em}{} 
\titleformat{\subsection}[block]{\large}{\thesubsection.}{1em}{} 


\usepackage{fancyhdr} 
\pagestyle{fancy} 
\fancyhead{} 
\fancyfoot{} 
\fancyhead[C]{Data Storytelling - \today} 
\fancyfoot[RO,LE]{\thepage} 


\usepackage{titling} 

%----------------------------------------------------------------------------------------
%	TILULOS
%----------------------------------------------------------------------------------------


\setlength{\droptitle}{-4\baselineskip} 

\pretitle{\begin{center}\Huge\bfseries} 
\posttitle{\end{center}} 
\title{Data Storytelling} 
\author{
	Valdivia Guzman, Alejandra Maria\\
	\and
	Pazos Alarcón, Christian Joshua\\
	\and
	Condori Quispe, Yhónn Joel\\
}
\date{\today} 
\renewcommand{\maketitlehookd}{
\selectlanguage{spanish} 
\begin{abstract}
\noindent 
Contar historias enseña. Toda buena narración es una buena enseñanza. Contar historias es una forma de enseñar
y aprender porque pide a los lectores u oyentes que sustituyan las explicaciones bien establecidas que se consideran
hechos por otras nuevas e inesperadas. Siempre aprendemos cosas nuevas cuando escuchamos buenas historias.
Las buenas historias lo hacen violando las expectativas y sorprendiendo al oyente o al lector. La sorpresa genera
suspense, lo que a su vez genera compromiso, que es un catalizador para el aprendizaje, lo que es aún más cierto
en el caso de las historias científicas y basadas en datos. En este artículo mostramos cómo entender la naturaleza
de las buenas historias centrándonos en la novedad que introducen y en la suposición que violan nos ayuda a realizar
un trabajo científico eficaz y a contar excelentes historias basadas en datos. Empezamos ofreciendo definiciones
de "historia", "narración" y "buenas historias".\\

A continuación, esbozamos una metodología para construir historias y ofrecemos un ejemplo ilustrativo de una historia
eficaz basada en datos de la historia de la medicina.
\end{abstract}
\selectlanguage{english} 
\begin{abstract}
\noindent 
Storytelling teaches. All good storytelling is good teaching. Storytelling is a form of teaching and learning because it
asks the readers or listeners to replace well-established explanations that are considered facts with new, unexpected
ones. We always learn new things when listening to good stories. Good stories do so by violating expectations and
surprising the listener or the reader. Surprise breeds suspense, which generates engagement, which is a catalyst for
learning – even truer for scientific and data-driven stories. In this paper we show how understanding the nature of
good stories by focusing on the novelty they introduce and assumption they violate helps us to do effective scientific
work and tell excellent data-based stories. We start by providing definitions of “story,” “storytelling,” and “good stories.”\\

We then outline a methodology for building stories and provide an illustrative example of an effective data-based story
from the history of medicine.
\end{abstract}
}

%----------------------------------------------------------------------------------------

\begin{document}

% Print the title
\maketitle

%----------------------------------------------------------------------------------------
%	Introduction
%----------------------------------------------------------------------------------------

\section{Introduction}

\lettrine[nindent=0em,lines=3]{O}ver the last five years, we’ve seen an explosion in excellent data storytelling.\\

A big part of this story has been the rise of data journalism. But we’ve also seen data storytelling from content
teams across brands, NGOs, universities, and more.\\

Some of the most acclaimed data stories have been produced by specialist data visualization and data science
teams — most famously at the New York Times and Five Thirty Eight. Many of these interactive stories are truly
stunning examples of what can be done with modern web browsers.\\

Here’s the problem, though: few content teams have the development resources of the Times. If all data storytelling
relied on a team of data scientists and web developers, then we wouldn’t see very many data-driven stories.\\

Happily, this isn’t the case. Nearly every media company in the world is publishing data stories, with data journalism
becoming an increasingly sought-after skill.\\

Even beyond media, we’re seeing stunning data visualisations from content marketing teams at businesses,
non-profits, universities, and more.


%----------------------------------------------------------------------------------------
%	State of Art
%----------------------------------------------------------------------------------------

\section{State of Art}

\subsection{Data Storytelling}
According to Knafilc’s idea, data itself is difficult to understand, but there are stories in data bringing it to life that
allow to communicate data in a much more efficient way. Data storytelling transforms data into a better form that
can support decision making. In the way of exploration of how to better present and deliver data, how to make the
process more vivid, convenient and plentiful\cite{zhang2018converging}.\\

Recent research in this domain has produced valuable concepts such as e.g. serious storytelling and cognitive big data.
Questions like the difference between serious storytelling and entertaining storytelling are addressed. It also rethinks, if
some methods and theories of traditional storytelling can support serious storytelling; and which domains or disciplines
should be considered. These concepts are vital how data storytelling supporst cognition and human activities\cite{zhang2018converging}.

\subsection{Visual Data Storytelling Process}
The three main components—exploring data, making a story, and telling a story—are introduced with their respective
artifacts in a linear order. 

\begin{center}
	\includegraphics[width=7cm]{./images/components}
	Storytelling process: transforming data into visually shared stories\cite{lee2015more}.
\end{center}

\subsubsection{Exploring Data}
Exploring data involves the set of activities centered around exploring and analyzing data. Data is the raw material that constitutes
the source of the visual data story content. Pertinent data excerpts are collected through exploratory analysis. These may be simple
such as recorded data facts or steps from the analysis process. They may be more complex such as derived data insights, interesting
sets or sequences within the data, and/or process details and variations. They may include the first quick externalizations of the data
such as charts from spreadsheets or hand sketches made during the analysis. At this point in the process, this collection of excerpts
may or may not be tied to any specific visual representation. The result of data exploration when making a visual data story is a collection
of the chosen data excerpts\cite{lee2015more}.

\subsubsection{Making a Story}
To make a story the data excerpts gathered in step one need to be assembled into a storyline that is interesting, illuminating, and
compelling. The sequence plays a critical role in a story; the same set of excerpts can have impact or can fall flat. A significant part
of making a story is the process of constructing the storyline or plot. The activities involved are ordering, establishing logical
connections, developing flow, formulating a message, and creating the denouement. These activities that are often intertwined
may be achieved sequentially, simultaneously, or through multiple iterations. Furthermore, it is possible while developing the storyline
to find it necessary to go back to the explore data stage to gather more excerpts (e.g., insights or evidence). The final outcome of
this step in the process is the plot of a story which describes how the story pieces are related (e.g., in time, cause and effect,
patterns, etc.) and what they mean in an overall context\cite{lee2015more}. 

\subsubsection{Telling a Story}
Telling a story is the general process of materializing the abstract plot and delivering the story. It consists of the following
activities: building a presentation (i.e., creating story material with the chosen medium), sharing the story using the story
material, and finally receiving and handling the feedback from the audience. In the building phase of telling a story, a plot
and story pieces are taken and turned into story material. Story material is the materialization of each piece of this abstract
content through the development of visual representations, interactions, animations, annotations, or narration. For
example, story material could be one or more visualizations assembled in a slide deck, a video with narration, an infographic
presented on a poster, or a demo planned with an interactive system for the live presentation\cite{lee2015more}.

\subsection{Business Example}

\subsubsection{Spotify’s Wrapped}
Spotify is one of the most famous data storytellers out there. As a streaming platform, they collect endless amounts of data
on streaming behaviors.

Instead of using this only for their optimization efforts, Spotify also creates a year-end summary for listeners.

The result is shareable, story-worthy, and delightful. And they go all out with it, showing insights on genres, habits, and
making it something that is an experience all in itself\cite{matei2021data}.

Obviously, a lot of work goes into what Spotify does here - but they are thinking about data in a way that many businesses
don’t. And the result is something that boosts their visibility as customers reshare it\cite{Maskovasdata}.

\begin{center}
	\includegraphics[width=7cm]{./images/business_example}
\end{center}

%----------------------------------------------------------------------------------------
%	Conclusions
%----------------------------------------------------------------------------------------

\section{Conclusions}
You don’t have to be a professional designer to tell a great data story. So long as you pay attention to what matters most, you
can lean on your existing strengths (whether they’re analytical, communicative, creative, and/or relational) to create data
stories that will help your audience know better and do better.

%----------------------------------------------------------------------------------------
%	References
%----------------------------------------------------------------------------------------

\bibliographystyle{plain} 
\bibliography{references} 
\end{document}